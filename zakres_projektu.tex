\section{Zakres projektu}
  \subsection{Cel projektu}
    Celem projektu jest zbudowanie kompletnego systemu zarządzania przechowywalnią opon samochodowych.
    Na system składać się będą następujące elementy funkcjonalne:
    \begin{itemize}
      \item strona internetowa dostępna publicznie dla klientów przechowywalni,
      \item strona internetowa z dostępem limitowanym do zarządzania przechowywalnią,
      \item aplikacja na tablety, umożliwiająca rejestrowanie opon za pomocą kodu kreskowego.
    \end{itemize}
    Do elementów niewidocznych dla użytkowników należy wyróżnić:
    \begin{itemize}
      \item kontener serwletów udostępniający strony internetowe,
      \item serwer aplikacyjny udostępniający połączenie do bazy danych z poziomu strony internetowej oraz aplikacji na tablety,
      \item relacyjną bazę danych do przechowywania wszystkich danych.
    \end{itemize}
    Ideą projektu jest powiązanie powyżej wymienionych elementów w jeden spójny, bezpieczny i niezawodny system.
    Dostęp do wrażliwych części będzie wymagał uwierzytelnienia.
    Będzie również automatyczny backup na wypadek awarii.
    Backup będzie przechowywany na oddzielnym serwerze fizycznym.
    System będzie przyjazny i intuicyjny w obsłudze, a pracownicy będą przeszkoleni w jego używaniu.
    \subsubsection{Cele biznesowe}
      Wykonanie projektu rozwija wiele możliwości biznesowych przechowywalni.
      \begin{itemize}
        \item Usprawnienie dotychczasowego katalogowania opon, poprzez skanowanie kodów kreskowych opon.
        \item Uproszczenie zamawiania dostaw i odbiorów opon, przez dostarczenie tych możliwości przez interfejs WWW zamiast telefonicznego.
        \item Automatyzacja sprawdzania składu kompletów opon i przypisywania ich do właściciela.
        \item Zmniejszenie ilości zatrudnionych pracowników.
        \item Poprawienie wizerunku firmy, przez używania fachowego, dedykowanego oprogramowania.
      \end{itemize}
